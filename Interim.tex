\documentclass[12pt]{report}
\usepackage[english]{babel}
\usepackage{hyperref}
\usepackage[utf8]{inputenc}
\usepackage[table]{xcolor}
\usepackage{graphicx}
\usepackage{url}
\usepackage{cite}
\usepackage{amsmath}
\usepackage{graphicx}
\usepackage{parskip}
\usepackage{fancyhdr}
\usepackage{vmargin}
\usepackage{amssymb}% http://ctan.org/pkg/amssymb
\usepackage{pifont}% http://ctan.org/pkg/pifont

\graphicspath{ {/home/user/pictures/} }
\newcommand{\xmark}{\ding{53}}

\setmarginsrb{3 cm}{2.5 cm}{3 cm}{2.5 cm}{1 cm}{1.5 cm}{1 cm}{1.5 cm}

\pagestyle{fancy}
\fancyhf{}
\cfoot{\thepage}

\begin{document}

\begin{titlepage}
\centering
\vspace*{-4.0 cm}
\begin{center}    
\Large{PHYS400} \hfill \Large{2020-1}
\rule{\linewidth}{0.2 mm} \\[5.0 cm]
\end{center}
   	
\textsc{\Huge Interim Report}\\[5.0 cm]				
	
\begin{center}
\begin{tabular}{|l|l|}
\hline
\Large{Name of the Student} & Emin YÜKSEL
\\ \hline
\Large{Name of the Advisor} & Ali Murat GÜLER
\\ \hline
\Large{Project Title}       & DsTau 
\\ \hline
\end{tabular}
\end{center} 
\end{titlepage}

\tableofcontents
\pagebreak

\renewcommand{\thesection}{\arabic{section}}
\section{Introduction}

Measuring the cross-section of the neutrinos is crucial to understand Lepton Universality, which is an important principle for the Standard Model. In this experiment, proton-nucleus interaction used as a Tau neutrino source, but Tau neutrinos are not produced directly after scattering occurs. Instead after proton-nucleus interaction D-mesons ($D$, $D_s$ and $D0$) are produced, and $D_s$ meson decays into Tau and Tau neutrino (Figure \ref{fig:process}). Therefore, to calculate cross-section of the Tau neutrino cross-section of $D_s$ also must be known.~\cite{aoki_ariga_dmitrievsky_firu_forshaw_fukuda_gornushkin_guler_haiduc_2019}

\begin{figure}[htp]
\includegraphics[width=\linewidth]{dstauProcess.png}
\caption{Tau neutrino production diagram.~\cite[p.~4]{aoki_ariga_dmitrievsky_firu_forshaw_fukuda_gornushkin_guler_haiduc_2019}}
\label{fig:process}
\end{figure}

Neutrinos are leptons like electrons and, there are six leptons (electron, muon, tau and their neutrino counterparts). Since neutrinos are nearly massless and chargless unlike charged leptons, neutrino detection is a challenging process. Moreover, detecting Tau neutrino is even tougher than the others because most of the time Tau neutrinos produced by the oscillation of the other neutrinos. Therefore, cross-section measurement of the Tau neutrinos is considerably less accurate than the other neutrinos \ref{fig:cross}. So, the point of this experiment is decrease the uncertainty in the measurement of cross-section of Tau neutrinos.

\begin{figure}[htp]
\centering
\includegraphics[width = 6 cm]{LeptonCross.png}
\caption{Cross-Section of three neutrino flavors.~\cite[p.~2]{aoki_ariga_dmitrievsky_firu_forshaw_fukuda_gornushkin_guler_haiduc_2019}}
\label{fig:cross}
\end{figure}

The flight length and kink angle of $D_s$ are important characteristics that must be measured accurately. The detector of DsTau is designed to fulfill this property. DsTau experiment uses nuclear emulsion technology that presents sub-micron and a few mrad spatial and angular resolution respectively.

We have studies these two parameters by means of Monte Carlo simulation. Using the Pythia event generator, proton-nucleon interactions are generated and control plots are produced \ref{fig:result}.

\begin{figure}[htp]
\centering
\includegraphics[width = \linewidth]{pythiaOut.png}
\caption{Estimated flight length (left) and kink angle (right) measurements.~\cite[p.~5]{aoki_ariga_dmitrievsky_firu_forshaw_fukuda_gornushkin_guler_haiduc_2019}}
\label{fig:result}
\end{figure}


As shown in Figure~\ref{fig:detector} detector module consist of two parts. The first part is the decay module which, is made of tungsten and followed by the combination of emulsion film and plastic sheet. First, proton-nucleus scattering occurs in the tugstein plate then, the trace of the daughter particles captured by the emulsion films which, are made of silver halide crystals. After the experiment is completed, the trace of the particles can be scanned. The second part is made of emulsion film and lead and, the purpose of this layer is momentum measurement. To measure momentum MCS Figure \ref{fig:MCS} (Multi Coulomb Scattering) method can be used~\cite{Agafonova_2012}.

\begin{figure}[htp]
\centering
\includegraphics[width = \linewidth]{detector.png}
\caption{Schematic view of detector module.~\cite[p.~6]{aoki_ariga_dmitrievsky_firu_forshaw_fukuda_gornushkin_guler_haiduc_2019}}
\label{fig:detector}
\end{figure}

\begin{figure}[htp]
\centering
\includegraphics[width = \linewidth]{MCS.jpg}
\caption{Momentum measurement by MCS Method.~\cite[p.~6]{Agafonova_2012}}
\label{fig:MCS}
\end{figure}

To obtain the cross-section of the Tau neutrino in deep inelastic collisions like in the DONuT experiment~\cite{kodama_ushida_andreopoulos_saoulidou_tzanakos_yager_baller_boehnlein_freeman_lundberg_2016} total cross-section formula will be used. 

\begin{equation}
\sigma_{\nu_\tau} = \sigma^{const}_{\nu_\tau} \cdot E \cdot K(E)
\end{equation}

Where $\sigma^{const}_{\nu_\tau}$ is energy independent cross-section, $E$ is Energy of the neutrino and, $K(E)$ is Kinematic effect due to lepton mass. Because of the uncertainty in the $D_s$ differential cross-section, $\sigma^{const}_{\nu_\tau}$ value is can be expressed in terms of n (longitudinal dependence parameter).

\begin{equation}
\sigma^{const}_{\nu_\tau} = 7.5(0.335n^{1.52}) \times 10^{-40} cm^{2} GeV^{-1}
\end{equation}

Moreover, by using $n = 6.1$ which, is obtained from PYTHIA simulation~\cite{sjostrand_mrenna_skands_2007} for DONuT experiment, $\sigma^{const}_{\nu_\tau}$ will be as

\begin{equation}
\sigma^{const}_{\nu_\tau} = (0.39 \pm 0.13 \pm 0.13) \times 10^{-38} cm^{2} GeV^{-1}
\end{equation}

\newpage

\section{Project Summary}

\begin{table}[htp]
\centering
\rowcolors{1}{black!0}{black!15!blue!5}
\begin{tabular}{ | c || c | c | c | c | c | c | c | c | c | c | c | c | c | }
\hline
\multicolumn{14}{|c|}{Weeks}\\
\hline
\hline
 & 2 & 3 & 4 & 5 & 6 & 7 & 8 & 9 & 10 & 11 & 12 & 13 & 14 \\
\hline
\hyperref[sec:WP1]{WP 1} & \xmark & \xmark & \xmark & & & & & & & & & & \\
\hline
\hyperref[sec:WP2]{WP 2} & & & \xmark & \xmark & \xmark & & & & & & & & \\
\hline
\hyperref[sec:WP3]{WP 3} & & & & & \xmark & \xmark & \xmark & \xmark & & & & & \\
\hline
\hyperref[sec:WP4]{WP 4} & & & & & & & & \xmark & \xmark & \xmark & & & \\
\hline
WP 5 & & & & & & & & & & \xmark & \xmark & \xmark & \xmark \\
\hline
WP 6 & & & & & & & & & & & \xmark & \xmark & \\
\hline

\end{tabular}
\end{table}

\begin{table}[htp]
\centering
\begin{tabular}{ | c | p{10cm} | }
\hline
\multicolumn{2}{ | c | }{Description of Work Packages} \\
\hline\hline
WP 1 & {\small Literature Search} \\
\hline
WP 2 & {\small Learn root sofware, analyze pythia/fluka monte carlo root files} \\
\hline
WP 3 & {\small Learn TMVA software, find discriminator parameters for signal events} \\
\hline
WP 4 & {\small Using machine learning techniques seperate signal from background} \\
\hline
WP 5 & {\small Prepare the written report} \\
\hline
WP 6 & {\small Prepare the poster} \\
\hline
\end{tabular}
\end{table}


\subsection{Workpackage 1}
\label{sec:WP1}
Literature search about DsTau experiment concluded. Since the experiment was conducted before~\cite{aoki_ariga_dmitrievsky_firu_forshaw_fukuda_gornushkin_guler_haiduc_2019}, used as a guide while reconstructing that experiment. Studied deep inelastic collisions which, are essential for the experiment. Also, done research about experiment instruments that affect the efficiency and accuracy of the measurements. Since the kink angle of particles in this experiment is very narrow accuracy of the emulsion films must be high. Also, since tau neutrino is difficult to detect, instruments must be efficient. Moreover, to determine particle ID and to distinguish background from the signal, some methods must be applied, which also affects efficiency. That is why efficiency is crucial.

\subsection{Workpackage 2}
\label{sec:WP2}
For the computational part of the experiment worked at PYTHIA to reconstruct the experiment on the computer. 

By using event generator the (PYTHIA 8186) DsTau decay simulated. PYTHIA’s output contains the momentum of the particle and the coordinates of the decay position. Furthermore, by using momenta of the particles in each event, and with the help of the cosine rule, kink angle of each particle is calculated (Figure \ref{fig:Kink}). Also, flight length can be calculated straightforwardly by subtracting the decay position of the particle from the first position that particle produced and get the magnitude of that displacement vector (Figure \ref{fig:Flight}).

\begin{figure}[htp]

\centering
\includegraphics[width = \linewidth]{KinkAngle.png}
\caption{Kink angle of $\tau$ and $D_s$}
\label{fig:Kink}

\includegraphics[width = \linewidth]{FlightLength.png}
\caption{Flight length of $\tau$ and $D_s$}
\label{fig:Flight}

\end{figure}

Moreover, PYTHIA is not the only event generator used during the experiment. To comparison, previously produced DsTau root data that generated by FLUKA event generator modified and compared with the PYTHIA’s output. Also, worked at ROOT to read FLUKA data and use TMVA~\cite{hoecker}.

\subsection{Workpackage 3}
\label{sec:WP3}
PYTHIA constructing events one by one and finding the measurement of each particle with the Monte-Carlo method. Therefore, in a PYTHIA output file, one can easily follow events particle by particle. On the other hand, FLUKA simulating tracks and sending protons in it like in real life. So, in the output file, one can only see the histogram of specific parameters for all events. Therefore, parameters must be separated for desired particles. To do that, TMVA can be used. Firstly, differentiating parameters (Kinematics, Kink Angle, etc.) between particles must be found. Then, with the help of machine learning, one can obtain the result. Currently, working on this part and made attempts to find separated data.

\subsection{Workpackage 4}
\label{sec:WP4}
In real life, there is also background process that mimics the signal in the experiments. To eliminate the background data TMVA can be used again. It is similar to the previous work package. Firstly, some parameters must be chosen to distinguish background data from the signal. Then, by using machine learning, the background data can be separated from the signal data.

\newpage

\section{Future Outlook}

After the particle separation process in work package 3 is completed, the process will continue with work package 4. The plan here to find the parameters to train the model, then start working at TMVA. For now, the two parameters that are considered to be here are the kink angle and flight length. Moreover, before concluding the project, the longitudinal dependence parameters with PYTHIA will be calculated and compared with the empirical values found in DONuT.

\bibliography{dstau}{}
\bibliographystyle{plain}
\end{document}
